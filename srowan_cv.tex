%!TEX TS-program = xelatex
%!TEX encoding = UTF-8 Unicode
% Awesome CV LaTeX Template for CV/Resume
%
% This template has been downloaded from:
% https://github.com/posquit0/Awesome-CV
%
% Author:
% Claud D. Park <posquit0.bj@gmail.com>
% http://www.posquit0.com
%
%
% Adapted to be an Rmarkdown template by Mitchell O'Hara-Wild
% 23 November 2018
%
% Template license:
% CC BY-SA 4.0 (https://creativecommons.org/licenses/by-sa/4.0/)
%
%-------------------------------------------------------------------------------
% CONFIGURATIONS
%-------------------------------------------------------------------------------
% A4 paper size by default, use 'letterpaper' for US letter
\documentclass[11pt,a4paper,]{sr-awesome-cv}

% Configure page margins with geometry
\usepackage{geometry}
\geometry{left=1.4cm, top=.8cm, right=1.4cm, bottom=1.8cm, footskip=.5cm}


% Specify the location of the included fonts
\fontdir[fonts/]

% Color for highlights
% Awesome Colors: awesome-emerald, awesome-skyblue, awesome-red, awesome-pink, awesome-orange
%                 awesome-nephritis, awesome-concrete, awesome-darknight

\definecolor{awesome}{HTML}{2698ba}

% Colors for text
% Uncomment if you would like to specify your own color
% \definecolor{darktext}{HTML}{414141}
% \definecolor{text}{HTML}{333333}
% \definecolor{graytext}{HTML}{5D5D5D}
% \definecolor{lighttext}{HTML}{999999}

% Set false if you don't want to highlight section with awesome color
\setbool{acvSectionColorHighlight}{true}

% If you would like to change the social information separator from a pipe (|) to something else
\renewcommand{\acvHeaderSocialSep}{\quad\textbar\quad}

\def\endfirstpage{\newpage}

%-------------------------------------------------------------------------------
%	PERSONAL INFORMATION
%	Comment any of the lines below if they are not required
%-------------------------------------------------------------------------------
% Available options: circle|rectangle,edge/noedge,left/right

\name{Sebastian}{Rowan}

\position{Ph.D Candidate in Civil and Environmental Engineering}
\address{University of New Hampshire, Durham, NH}

\pronouns{He/Him}
\email{\href{mailto:sebastian.rowan@unh.edu}{\nolinkurl{sebastian.rowan@unh.edu}}}
\homepage{sebastianrowan.github.io}
\github{sebastianrowan}
\linkedin{sebastian-rowan-72490170}

% \gitlab{gitlab-id}
% \stackoverflow{SO-id}{SO-name}
% \skype{skype-id}
% \reddit{reddit-id}

\quote{I am Ph.D.~candidate in civil and environmental engineering
researching the impacts of flood events and climate change on people and
communities. The goal of my research, is to develop a more comprehensive
understanding of the risks posed by floods to enable the development of
mitigation efforts that prioritize long-term sustainability and
community well-being rather than maximizing financial return on
investment.}

\usepackage{booktabs}

\providecommand{\tightlist}{%
	\setlength{\itemsep}{0pt}\setlength{\parskip}{0pt}}

%------------------------------------------------------------------------------



% Pandoc CSL macros
\newlength{\cslhangindent}
\setlength{\cslhangindent}{1.5em}
\newlength{\csllabelwidth}
\setlength{\csllabelwidth}{2em}
\newenvironment{CSLReferences}[2] % #1 hanging-ident, #2 entry spacing
 {% don't indent paragraphs
  \setlength{\parindent}{0pt}
  % turn on hanging indent if param 1 is 1
  \ifodd #1 \everypar{\setlength{\hangindent}{\cslhangindent}}\ignorespaces\fi
  % set entry spacing
  \ifnum #2 > 0
  \setlength{\parskip}{#2\baselineskip}
  \fi
 }%
 {}
\usepackage{calc}
\newcommand{\CSLBlock}[1]{#1\hfill\break}
\newcommand{\CSLLeftMargin}[1]{\parbox[t]{\csllabelwidth}{\honortitlestyle{#1}}}
\newcommand{\CSLRightInline}[1]{\parbox[t]{\linewidth - \csllabelwidth}{\honordatestyle{#1}}}
\newcommand{\CSLIndent}[1]{\hspace{\cslhangindent}#1}

\begin{document}

% Print the header with above personal informations
% Give optional argument to change alignment(C: center, L: left, R: right)
\makecvheader

% Print the footer with 3 arguments(<left>, <center>, <right>)
% Leave any of these blank if they are not needed
% 2019-02-14 Chris Umphlett - add flexibility to the document name in footer, rather than have it be static Curriculum Vitae
\makecvfooter
  {September 2023}
    {Sebastian Rowan~~~·~~~Curriculum Vitae}
  {\thepage}


%-------------------------------------------------------------------------------
%	CV/RESUME CONTENT
%	Each section is imported separately, open each file in turn to modify content
%------------------------------------------------------------------------------



\hypertarget{education}{%
\section{\faMortarBoard~Education}\label{education}}

\begin{cventries}
    \cventry{Ph.D. Candidate, Civil and Environmental Engineering}{University of New Hampshire}{Durham, New Hampshire}{2024 (Expected)}{\begin{cvitems}
\item Dissertation Title: Assessment of the carbon footprint of flood damages and flood risk management strategies.
\item Advisor: Dr. Weiwei Mo
\end{cvitems}}
    \cventry{B.S. Environmental Engineering}{University of New Hampshire}{Durham, New Hampshire}{2016}{}\vspace{-4.0mm}
\end{cventries}

\hypertarget{experience}{%
\section{\faBriefcase~Experience}\label{experience}}

\begin{cventries}
    \cventry{ORISE Graduate Research Fellow}{U.S. Army Corps of Engineers, Engineer Research Development Center}{Vicksburg, MS (Remote)}{Sep. 2020 - Present}{\begin{cvitems}
\item Contributed to research efforts to quantify impacts of floods not typically included in cost-benefit analyses for flood risk management projects.
\item Lead systematic literature review and metasummary to identify risk factors for mental health impacts of floods
\item Contributed to Tier 1 Other Social Effects/Environmental Justice Analysis for the USACE New York/New Jersey Harbors and Tributaries Coastal Storm Risk Management Feasibility Study
\item Contributed to social vulnerability analysis of future flooding in the Mississippi River Valley
\end{cvitems}}
    \cventry{Research Assistant}{University of New Hampshire}{Durham, NH}{Fall 2018, Spring 2020}{\begin{cvitems}
\item Resilient Bridge Planning in Mozambique - Bridge Failure Risk from Flooding and Climate Change
\item PI: Dr. Kyle Kwiatkowski
\end{cvitems}}
    \cventry{Teaching Assisstant}{University of New Hampshire}{Durham, NH}{2019 - 2020}{\begin{cvitems}
\item CEE 705: Introduction to Sustainable Engineering (Fall 2019, Fall 2020)
\item CEE 502: Project Engineering (Spring 2019)
\end{cvitems}}
    \cventry{Civil Engineer I-II}{New Hampshire Department of Transportation, Bureau of Planning and Community Assistance}{Concord, NH}{2016 - 2018}{\begin{cvitems}
\item Contributed to the development of statewide asset management system for culvert and closed drainage systems in partnership with UNH Technology Transfer Center/SADES.
\end{cvitems}}
    \cventry{10 Gigabit Ethernet Technician}{University of New Hampshire InterOperability Laboratory}{Durham, NH}{2014-2017}{}\vspace{-4.0mm}
    \cventry{Summer Intern}{New Hampshire Department of Environmental Services, Air Resources Division}{Concord, NH}{2015}{}\vspace{-4.0mm}
\end{cventries}

\hypertarget{open-source-projects}{%
\section{\faGithub~Open Source Projects}\label{open-source-projects}}

\begin{cventries}
    \cventry{sviBuildr}{}{}{Active}{\begin{cvitems}
\item An R package that allows users to download or construct the CDC's Social Vulnerability Index as a tidyverse or simple features data frame.
\item Enables greater flexibility in region selection for SVI analyses than is possible with state- or national-level datasets hosted by CDC.
\end{cvitems}}
    \cventry{NSI Data QGIS Plugin}{}{}{Active}{\begin{cvitems}
\item A basic plugin for QGIS that downloads data from the USACE National Structures Inventory for a specified region and adds it to a map.
\end{cvitems}}
    \cventry{useeio\_py}{}{}{Active}{\begin{cvitems}
\item A Python translation of the USEPA's useeior R package for building and using USEEIO models for life cycle analysis.
\end{cvitems}}
\end{cventries}

\hypertarget{publications}{%
\section{\faFileText~Publications}\label{publications}}

\footnotesize

\begin{itemize}
\item
  \textbf{Rowan, S.}, Yeates, E., Wells, E., Murray, B., \& Pinigina, E.
  (2023). Risk Factors for Mental Health Impacts From Floods: A
  Systematic Literature Review and Metasummary. Natural Hazards
  {[}Manuscript Submitted for Publication{]}.
\item
  Galaitsi, S., Kurth, M., \textbf{Rowan, S.}, Yeates, E., \&
  Kalaidjian, E. (2022). New York---New Jersey Harbor and Tributaries
  Coastal Storm Risk Management Feasibility Study---Tier 1 Other Social
  Effects/Environmental Justice Analysis. U.S. Army Corps of Engineers
  New York District.
  \url{https://www.nan.usace.army.mil/Portals/37/Appendix\%20A12_Tier\%201\%20OSE_EJ_HATS.pdf}
\item
  Seigerman, C. K., McKay, S. K., Basilio, R., Biesel, S. A.,
  Hallemeier, J., Mansur, A. V., Piercy, C., \textbf{Rowan, S.}, Ubiali,
  B., Yeates, E., \& Nelson, D. R. (2023). Operationalizing equity for
  integrated water resources management. JAWRA Journal of the American
  Water Resources Association, 59(2), 281--298.
  \url{https://doi.org/10.1111/1752-1688.13086}
\item
  \textbf{Rowan, S.}, \& Kwiatkowski, K. (2020). Assessing the
  Relationship Between Social Vulnerability, Social Capital, and Housing
  Resilience. Sustainability, 12(18), 7718.
  \url{https://doi.org/10.3390/su12187718}
\end{itemize}

\normalsize

\hypertarget{presentations}{%
\section{\faUser~Presentations}\label{presentations}}

\footnotesize

\begin{itemize}
\item
  \textbf{Rowan, S.}, Yeates, E., Mo, W. Estimating the Greenhouse Gas
  Emissions of Flood Damages. \emph{AEESP Research \& Education
  Conference; June 2023; Boston, MA.} Poster.
\item
  \textbf{Rowan, S.}, Yeates, E. Predicting the Mental Health Impacts of
  Floods. \emph{47th Annual Natural Hazards Research and Applications
  Workshop; July 2022; Virtual.} Poster.
\item
  \textbf{Rowan, S.}, Kwiatkowski, K. Assessing the Relationship Between
  Social Vulnerability, Social Capital, and Housing Resilience.
  \emph{46th Annual Natural Hazards Research and Applications Workshop;
  July 2020; Virtual.} Poster.
\item
  \textbf{Rowan, S.}, Kwiatkowski, K., Qiao, Y. Resilient Bridge
  Planning in Mozambique: Bridge Failure Risk from Flooding and Climate
  Change. \emph{2nd International Conference on Transportation System
  Resilience to Natural Hazards and Extreme Weather Events (TR2019);
  November 2019; Washington, D.C.} Presentation.
\end{itemize}

\normalsize



\end{document}
